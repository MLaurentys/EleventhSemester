%%%%%%%%%%%%%%%%%%%%%%%%%%%%%%%%%%%%%%%%%%%%%%%%%%%%%%%%%%%%%%%%%%%%%%%%%%%%%%%%
%%%%%%%%%%%%%%%%%%%%%%%%%%%% OUTROS PACOTES ÚTEIS %%%%%%%%%%%%%%%%%%%%%%%%%%%%%%
%%%%%%%%%%%%%%%%%%%%%%%%%%%%%%%%%%%%%%%%%%%%%%%%%%%%%%%%%%%%%%%%%%%%%%%%%%%%%%%%

% Você provavelmente vai querer ler a documentação de alguns destes pacotes
% para personalizar algum aspecto do trabalho ou usar algum recurso específico.

% A classe Book inclui o comando \appendix, que (obviamente) permite inserir
% apêndices no documento. No entanto, não há suporte similar para anexos. Esta
% package (que não é padrão do LaTeX, foi criada para este modelo) define o
% comando \annex. Ela deve ser carregada depois de hyperref.
\dowithsubdir{extras/}{\usepackage{annex}}

% Para inserir separações no texto que não correspondem a seções com um nome
% definido, é comum usar um ornamento ou florão (em inglês e francês: fleuron).
% Esta package define o comando \frufru que insere um florão desse tipo.
\dowithsubdir{extras/}{\usepackage{frufru}}

% Formatação personalizada das listas "itemize", "enumerate" e
% "description", além de permitir criar novos tipos de listas.
% Com a opção "inline", a package define os novos ambientes "itemize*",
% "description*" e "enumerate*", que fazem os itens da lista como
% parte de um único parágrafo. Como ela causa problemas com
% beamer, apenas a carregamos se não estivermos usando beamer.
\makeatletter
\@ifclassloaded{beamer}
  {}
  {\usepackage[inline]{enumitem}}
\makeatother

% Sublinhado e outras formas de realce de texto
\usepackage{soul}
\usepackage{soulutf8}

% Melhorias e personalização do sublinhado com soul (comando \ul)

% Distância e largura do sublinhado
\setul{1.4pt}{.5pt}

% btul -> "Better Underline" (https://alexwlchan.net/2017/10/latex-underlines/ )
% Sublinhado sem cruzar as linhas descendentes dos caracteres
\usepackage[outline]{contour}
\contourlength{1.1pt}
\let\ORIGul\ul
\newcommand{\btul}[2][white]{%
  \contourlength{1.1pt}%
  \setul{1.4pt}{.5pt}%
  \ORIGul{{\phantom{#2}}}% Faz o sublinhado; precisa das chaves adicionais!
  \llap{\contour{#1}{#2}}% Escreve o texto com fundo branco/colorido
}

% Não faço ideia se isto pode ter efeitos colaterais
% indesejáveis, então melhor deixar desabilitado
%\let\ul\btul
%\let\underline\btul

% Notação bra-ket
%\usepackage{braket}

% Vários recursos para apresentação de números e grandezas (unidades, notação
% científica, melhor apresentação de números longos etc.), além de permitir
% alinhar números em tabelas pelo ponto decimal (como a package dcolumn)
% através do tipo de coluna "S". Por exemplo, \SI{10}{\hertz} ou
% \num[round-mode=places,round-precision=2]{3.1415926}.
\usepackage[binary-units]{siunitx}
\sisetup{
  mode=text,
  round-mode=places,
}

\providetranslation[to=Portuguese]{to (numerical range)}{a}
\providetranslation[to=Portuguese]{and}{e}
\addto\extrasbrazil{\sisetup{output-decimal-marker = {,}}}

% Citações melhores; se você pretende fazer citações de textos
% relativamente extensos, vale a pena ler a documentação. biblatex
% utiliza recursos deste pacote.
\usepackage{csquotes}

\usepackage{url}
% URL com fonte sem serifa ao invés de teletype
\urlstyle{sf}

% Permite inserir comentários, muito bom durante a escrita do texto;
% você também pode se interessar pela package pdfcomment.
\usepackage[textsize=scriptsize,colorinlistoftodos,textwidth=2.5cm]{todonotes}
\presetkeys{todonotes}{color=orange!40!white}{}

% Comando para fazer notas com highlight no texto correspondente:
% \hltodo[texto][opções]{comentário}
\makeatletter
\if@todonotes@disabled
  \NewDocumentCommand{\hltodo}{O{} O{} +m}{#1}
\else
  \NewDocumentCommand{\hltodo}{O{} O{} +m}{
    \ifstrempty{#1}{}{\texthl{#1}}%
    \todo[#2]{#3}{}%
  }
\fi
\makeatother

% Vamos reduzir o espaçamento entre linhas nas notas/comentários
\makeatletter
\xpatchcmd{\@todo}
  {\renewcommand{\@todonotes@text}{#2}}
  {\renewcommand{\@todonotes@text}{\begin{spacing}{0.5}#2\end{spacing}}}
  {}
  {}
\makeatother

% Símbolos adicionais, como \celsius, \ohm, \perthousand etc.
%\usepackage{gensymb}

% Símbolos adicionais, como \textrightarrow, \texteuro etc.
\usepackage{textcomp}

% Permite criar listas como glossários, listas de abreviaturas etc.
% https://en.wikibooks.org/wiki/LaTeX/Glossary
%\usepackage{glossaries}

% Permite formatar texto em colunas
\usepackage{multicol}

% Gantt charts; útil para fazer o cronograma para o exame de
% qualificação, por exemplo.
\usepackage{pgfgantt}

% Na versão 5 do pacote pgfgantt, a opção "compress calendar"
% deixou de existir, sendo substituída por "time slot unit=month".
% Aqui, um truque para funcionar com ambas as versões.
\makeatletter
\@ifpackagelater{pgfgantt}{2018/01/01}
  {\ganttset{time slot unit=month}}
  {\ganttset{compress calendar}}
\makeatother

% Estes parâmetros definem a aparência das gantt charts e variam
% em função da fonte do documento.
\ganttset{
    time slot format=isodate-yearmonth,
    vgrid,
    x unit=1.7em,
    y unit title=3ex,
    y unit chart=4ex,
    % O "strut" é necessário para alinhar o baseline dos nomes dos meses
    title label font=\strut\footnotesize,
    group label font=\footnotesize\bfseries,
    bar label font=\footnotesize,
    milestone label font=\footnotesize\itshape,
    % "align=right" é necessário para \ganttalignnewline funcionar
    group label node/.append style={align=right},
    bar label node/.append style={align=right},
    milestone label node/.append style={align=right},
    group incomplete/.append style={fill=black!50},
    bar/.append style={fill=black!25,draw=black},
    bar incomplete/.append style={fill=white,draw=black},
    % Não é preciso imprimir "0%"
    progress label text=\ifnumequal{#1}{0}{}{(#1\%)},
    % Formato e tamanho dos elementos
    title height=.9,
    group top shift=.4,
    group left shift=0,
    group right shift=0,
    group peaks tip position=0,
    group peaks width=.2,
    group peaks height=.3,
    milestone height=.4,
    milestone top shift=.4,
    milestone left shift=.8,
    milestone right shift=.2,
}

% Em inglês, tanto o nome completo quanto a abreviação do mês de maio
% são "May"; por conta disso, na tradução em português LaTeX erra a
% abreviação. Como talvez usemos o nome inteiro do mês em outro lugar,
% ao invés de forçar a tradução para "Mai" globalmente, fazemos isso
% apenas em ganttchart.
\AtBeginEnvironment{ganttchart}{\deftranslation[to=Portuguese]{May}{Mai}}

% Ilustrações, diagramas, gráficos etc. criados diretamente em LaTeX.
% Também é útil se você quiser importar gráficos gerados com GnuPlot.
\usepackage{tikz}

% Gráficos gerados diretamente em LaTeX; é possível usar tikz para
% isso também.
\usepackage{pgfplots}
% sobre níveis de compatibilidade do pgfplots, veja
% https://tex.stackexchange.com/a/81912/183146
%\pgfplotsset{compat=1.14} % TeXLive 2016
%\pgfplotsset{compat=1.15} % TeXLive 2017
%\pgfplotsset{compat=1.16} % TeXLive 2019
\pgfplotsset{compat=newest}

% Importação direta de arquivos gerados por gnuplot com o
% driver/terminal "lua tikz"; esta package não faz parte da
% instalação padrão do LaTeX, mas sim do gnuplot.
%\usepackage{gnuplot-lua-tikz}

% O formato pdf permite anexar arquivos ao documento, que aparecem
% na página como ícones "clicáveis"; esta package implementa esse
% recurso em LaTeX.
%\usepackage{attachfile}

% Os comandos \TeX e \LaTeX são nativos do LaTeX; esta package acrescenta os
% comandos \XeLaTeX e \LuaLaTeX. Você provavelmente não precisa desse recurso
% e, portanto, pode removê-la.
\usepackage{metalogo}
\providecommand{\ConTeXt}{\textsc{Con\TeX{}t}}
% Outros logos da família TeX; você também pode remover estas linhas:
\usepackage{hologo}
\renewcommand{\ConTeXt}{\hologo{ConTeXt}}
