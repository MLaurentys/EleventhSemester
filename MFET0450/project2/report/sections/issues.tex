The System as implemented in the simulation and described section 2 will resist some errors resulting prom emergency stops and power loss. The current state of the system, however, is not safe for all scenarios, including resistance to personnel mistakes and parts failure. This section brings some of them into light.

Due to the high speed, the bottle conveyor should be kept in a restricted area when operating. Although there is an E-Stop rope around the conveyor, since there are no proximity sensors the speed might be high enough to cause damage to personnel in the restricted area before the E-Stop is actuated. However, this injuries are not major nor fatal, and by making the zone where the System is installed restricted, the accidents will be rare.

If it is impossible to restrict the area where the feeders and fillers are installed, then physical barriers should be added is all reachable parts. In this cases, using proximity sensors is a good addition to system safety. For example, if the part holding the bottles fails, a bottle will fall in a very high speed. If it is in an accessible area and there are no barriers, it might hurt personnel.

Additionally, the system does not take into account misuse of the equipment. As from the PLC program, the packaging points cannot be accessed when the machine is on. packaging is only allowed when conveyor power is off because there is a risk of having other products coming down from the conveyor in high speed. However, since there are no physical blocks that becomes a risk in the physical implementation.

The last featured problem is with material handling and monitoring parts failure. If the liquid feeder fails and does not detect it is out of liquid, the system will continue running normally. While not a risk to personnel, the result would be packaged empty bottles, which is a big problem. The problem is there are no feedback sensors. The system only uses timers and trusts every subsystem works all the time. This is not realistic but is a decision taken to avoid unnecessary complication in the earlier stages of development. If moving to a physical scenario, feedback sensors would be necessary to monitor the system.