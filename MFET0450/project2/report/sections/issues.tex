The system as implemented in the simulation and described section 2 will resist some errors resulting prom emergency stops and power loss. The current state of the system, however, is not safe for all scenarios, including resistance to personnel mistakes, power loss and package jams. This section brings some of them into light.

The conveyor is intended to operate in accessible places as it does not present any major or fatal injury risk. Adding physical barriers might reduce the frequency of injuries but that happens in exchange of productivity gains. It is a decision that may vary depending on the specification of sizes of products or speed of conveyors. However, when looking at production environments, which typically do not have barriers, a decision to not use them was made. The main risk is when a product is redirected by the belt diverter, a PLC or device errors happens, and the product ends up falling, maybe in some employee.

However, the current version of the system does not take into account misuse of the equipment. As from the PLC program, the collection points cannot be accessed when the machine is on due to the risk of having other products coming down from the conveyor in high speed. However, since there are no physical blocks that becomes a risk.

A second issue, that was brought to light since section two is that not all problems related to power loss were handled. If an item is high speed is coming down a line and the automated gravity rollers used do not have power to break their movement, the product may reach the end of the line, falling in the ground and breaking or maybe even hitting someone.

The third problem is that the system does not handle hardware issues, by using additional sensors for example, nor does it fail safe. When a jam happens, for example, it goes unnoticed by the system and this can lead to multiple issues, from braking products, to breaking sensors and even hurting people. While there were attempts to solve safety issues, it is very clear they persist. 